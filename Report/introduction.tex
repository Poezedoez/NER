\section{Introduction} \label{sec:introduction}
Named Entity Recognition(NER) is a task of automatically finding Named Entities(NE's) such as persons, locations and organizations in text documents, as well as disambiguating between said types. Different methods have been tried over the last two decades, most of them language dependent. NER is generally applied to a specific domain for which the task is optimized, such as finding chemical NE's \cite{rocktaschel2012chemspot}. 

In this paper I want to investigate NER in the domain of Dutch lobby documents. Lobbying can be defined as the act of attempting to influence decisions made by officials in a government, most often legislators or members of regulatory agencies. Lobby documents are for example: resolutions, \textit{kamervragen}, which is a Dutch term for describing questions asked by De Eerste Kamer (the Senate) and De Tweede Kamer (the House of Representatives) to the government, letters to the government and more. A proposed goal system that makes use of the NE's that have been found, allows parties of interest to 'subscribe' to particular entities. These parties will be notified of any new lobby documents containing these entities. 

A relatively new parser, Frog\footnote{https://languagemachines.github.io/frog/}, can do a multitude of things in analyzing a text. So far there haven't been any scientific evaluations of the parser, therefore in this paper I'm going to evaluate the part-of-speech(POS) assignment and the NER-task of this particular parser for the domain of lobby documents. Although this paper is focused on the NER task, POS assignment has proven to be an important feature in finding NE's, as is why the CoNLL-2002 dataset, which was used in the CoNLL2002 shared task\footnote{http://www.cnts.ua.ac.be/conll2002/ner/} of NER , contains these assigments for each word token. May the results of the NER task performed by Frog turn out sub-par, while the POS accuracy also turned out below average, the latter could be an explanation.

\begin{enumerate}
    \item What is the best way of finding Named Entities in lobby documents?
    \begin{enumerate}
        \item   How does Frog parser perform in finding NE's in lobby documents?  On which areas does Frog perform well? On which areas does it not do well?
        \item How can Frog be improved for the given task?
    \end{enumerate}
\end{enumerate}


\paragraph{Overview of thesis}
At first I want to look at research that has been done on this topic, and take a more in-depth look at the Frog parser and its components. Then I will describe my approach of evaluating the Frog parser on lobby documents and the way of doing the error analysis. This approach is followed by the results, which will form a conclusion about how well Frog is currently suited for the main task as well as how Frog can be improved for this task. A discussion of approach will conclude this paper.

