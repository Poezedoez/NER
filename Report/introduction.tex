\section{Introduction} \label{sec:introduction}
Named Entity Recognition(NER) is a technique used in Information Retrieval which can be described as automatically finding Named Entities(NE's) such as persons, locations and organizations in text documents, as well as disambiguating between said types. Different methods have been tried over the last two decades, most of them language dependent, due to the fact that language features are used as indicators, yet grammar and vocabulary differ greatly across languages. While near-human performing NER-systems have already been developed for English, this does not appear to be the case for Dutch. The reason for this being that most natural language applications are confined to the language in the field of study, for which English is predominant.  

Although progress has been made, especially for English, several encountered problems are not paired with unanimously accepted solutions. 
One might for example reason that a large gazetteer, which is actively updated, could be used in the retrieval of entities. While good results have been found using this approach, it, solely, cannot be carried out into large scale applications, because the completeness of the gazetteer and the application run time are inversely proportional to each other. While small gazetteers will affect recall negatively, comprehensive gazetteers require more time to search through, even with hashing techniques. 
Secondly, ambiguity introduces itself as the principal difficulty for type assignment: An entity can be of different role, albeit completely identical in appearance. Humans are able to quite easily disambiguate an entity, even though elucidation of choice can be difficult. Therefore, computer systems are set up with an arduous task that automates this decision process.

NER is generally applied to a specific domain for which the task is optimized, such as finding chemical NE's \cite{rocktaschel2012chemspot}. 
In this paper NER is applied in the domain of Dutch parliamentary items, which are for example resolutions proposed or letters written by De Eerste Kamer (the Senate) and De Tweede Kamer (the House of Representatives) to the government. A suggested goal system that utilizes the acquired NE's, notifies parties of interest of any new, relevant parliamentary items.
The answer to the question of how to effectively acquire these NE's with corresponding type for the suggested system, will be sought for.
This is an improvement to a baseline string search, because unknown strings are not retrieved. Nor do the retrieved strings contain meaning, as they hold no type information.

A relatively new parser, Frog\footnote{\url{https://languagemachines.github.io/frog/}}, that has been maintained, can apply a multitude of natural language processing techniques on a text. Frog will be examined thoroughly in section \ref{sec:rel}. Hitherto no scientific evaluations of the latest iteration of the system have been published. For this reason an evaluation of  the NER-task of Frog for the domain of parliamentary items has been performed and results are compared to those achieved on the CoNLL-2002 dataset, which was used in the CoNLL-2002 shared task\footnote{\url{http://www.cnts.ua.ac.be/conll2002/ner/}}. Subsequently, a method will be shown for the improvement of entity \textit{type} assignment using majority voting and a parliamentary gazetteer.

\subsection{Overview of thesis}
At first research that has been done on this topic, is looked into, which will provide a more in-depth look at Frog and its NER-module. Secondly, the approach of evaluating Frog on parliamentary items is described, as well as the procedure of the error analysis and how improvement of the entity type assignment was attempted. This approach is followed by the results, which will be used to form a conclusion about the suitability of Frog for the main task in addition to possible enhancements. A discussion of approach will conclude this paper.
